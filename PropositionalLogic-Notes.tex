\section{Dragon Acadmemy 2017-2018}\label{dragon-acadmemy-2017-2018}

\section{Computer Science G12}\label{computer-science-g12}

Remember I'm following the text I posted on
\href{http://msantos.sdf.org/G12/Term1/AnIntroductionToLogic-2016-MarkVLawson.pdf}{msantos.sdf.org/G12/}
.

These notes will be updated further in the coming days and can be found
on that same site, both in Markdown as well as PDF formats.

\section{Notes Tue 17 Oct 2017}\label{notes-tue-17-oct-2017}

\subsection{Boolean Algebra}\label{boolean-algebra}

\subsubsection{Connectives and their truth
tables}\label{connectives-and-their-truth-tables}

And, or, not, conditional, biconditional, xor:
\(\wedge,\,\vee,\,\urcorner,\,\rightarrow,\,\leftrightarrow,\,\oplus\)

Truth values: True/False, 0/1, T/\(\perp\)

\paragraph{Alternative notations}\label{alternative-notations}

While symbols listed above are the ones that are common in logic, some
computer science texts, or rather computer languages, may use other
symbols for them.

\begin{itemize}
\tightlist
\item
  Negation: NOT p, !p, \(\bar{p},\, p'\)
\item
  Conjunction: p AND q, \(p\cdot q,\quad p\,q\)
\item
  Disjunction: p OR q, \(p + q\)
\end{itemize}

\subsubsection{Boolean Laws}\label{boolean-laws}

\begin{enumerate}
\def\labelenumi{\arabic{enumi}.}
\tightlist
\item
  Commutativity
\item
  Associativity
\item
  Distributivity
\item
  Idempotence
\item
  De Morgan Laws
\end{enumerate}

\section{Notes of Tue Oct 31 2017.}\label{notes-of-tue-oct-31-2017.}

\subsection{Logic riddles:}\label{logic-riddles}

\begin{enumerate}
\def\labelenumi{\arabic{enumi}.}
\tightlist
\item
  A, B, C either Knights (allways tell truth) or Knaves (allways lie).
  You ask A, but she whispers and you don't understand it. B then says:
  ``She said she is a knave'', to which C immediately replies ``That's a
  lie''. What can you say about A,B, and C being either a knight or a
  knave?
\item
  A: ``Exactly one of us is a knave'', B: ``Exactly two of us is a
  knave'', C: ``all of us are knaves''. Same question.
\end{enumerate}

More on logic riddles can be found in e.g.

\begin{itemize}
\item
  The notes I'm basing this unit on. See link above.
\item
  The site
  \href{https://sites.google.com/site/newheiser/knightsandknaves}{Knight
  and Knaves}
\item
  Logicians have a peculiar sense of humor as reflected in the title of
  research papers like ``\emph{The Hardest Logic Puzzle Ever}'' by
  George Boolos or ``\emph{How to solve the hardest logic puzzle ever in
  two questions}'' by Gabriel Uzquiano. (And you thought they would only
  be dealing with \emph{serious stuff}, eh?)

  You can dig further into this topic in the corresponding
  \href{https://en.wikipedia.org/wiki/The_Hardest_Logic_Puzzle_Ever}{wikipedia
  page}. Be aware though, these are tricky problems, so don't desperate
  if you feel at lost -that said, if you find them trivial, let me know
  and I tell the authors!!
\end{itemize}

\subsection{SAT (Boolean SATisfiability)
problem:}\label{sat-boolean-satisfiability-problem}

\begin{itemize}
\tightlist
\item
  Example of a SAT problem: Which combination of truth values of p, q
  and r makes the following sentence true? \textasciitilde{}p \^{} (r
  \^{} (q v p) )
\item
  Easy understand complexity: Exponential complexity. Unsolved general
  case. P !=? NP
\end{itemize}

\subsection{Propositional Logic as a formal
system/language}\label{propositional-logic-as-a-formal-systemlanguage}

We will describe again here what we have seen on propositional logic,
but his time we will consider the topic as a \emph{(formal) language}.

In more general terms, a language is an example of what's called a
\emph{formal system}. Other examples of formal systems are for instance,
any \emph{axiomatic system}, e.g., Euclid's geometry could arguably be
considered the first axiomatic/formal system we know of.

One distinguishes two parts when formally stating what propositional
logic, and by extension, any (formal) language, is: Its \textbf{syntax}
and its \textbf{semantics}.

Note: What does the word semantic mean? where does it come from?
Semantics comes from greek
\(\sigma \eta \mu \alpha \nu \tau \iota \kappa \'o \varsigma\) (like
semanticos) and more or less translates into the word ``meaning''. Thus,
semantic as adjective relates to the meaning in language and logic.

\subsubsection{Syntax of Propositional
Logic}\label{syntax-of-propositional-logic}

\begin{enumerate}
\def\labelenumi{\arabic{enumi}.}
\tightlist
\item
  Alphabet of symbols:
  \(\Sigma=\{p_1,\,p_2,\dots ,\, (,),\urcorner ,\vee ,\wedge ,\rightarrow ,\leftrightarrow ,\oplus , \perp, T\}\)
\item
  Definition: \textbf{Atom} := Any of the symbols \(p_1,\,p_2,\dots\)

  \begin{itemize}
  \tightlist
  \item
    Note: Atoms are usually denoted by lowercase symbols from a natural
    language (e.g.~english) alphabet like \(p, q, r,\dots\) or variants
    as shown above (\(p_1,p_2\dots\)).
  \end{itemize}
\item
  Definition of a sentence using that alphabet \(\Sigma\): Sentences are
  strings containing only characters of the alphabet \(\Sigma\).
\item
  Rules defining a \emph{well-formed formula}:

  \begin{enumerate}
  \def\labelenumii{\arabic{enumii}.}
  \tightlist
  \item
    (WFF1) All atoms are wff
  \item
    (WFF2) if A and B are wff, the so are
    \((A), \urcorner A, A\vee B, A\wedge B, A\oplus B, A\rightarrow B, A\leftrightarrow B\).
  \item
    (WFF3) All wffs are constructed by repeated application of rules
    (WFF1) and (WFF2) a \emph{finite number of times}.
  \end{enumerate}
\end{enumerate}

Definition: All wff which are not an atom are called \emph{compound
statements}.

Definition: \textbf{Literal} := Any atom \(p\) or the symbol
\(\urcorner\) immediately followed by an atom \(p\). That is, either
\(p\) or \(\urcorner p\).

Definition: \textbf{Connective} := Any of the symbols of the alphabet
that either links together two or more wff or modifies one. Examples:
\(\urcorner ,\vee ,\wedge\dots\)

Definition: \textbf{Language based on alphabet \(\Sigma\)} := Is the
collection of all wff based on that alphabet.

Definition: A \textbf{grammar} is the set of rules that tells us what
the wff are.

\begin{itemize}
\tightlist
\item
  The grammar for propositional logic is given by the rules (WFF1),
  (WFF2) and (WFF3) above. This set of rules is an exampl of
  \emph{Backus-Knaur Form} (BNF).
\end{itemize}

The idea of a language as a formal system, i.e., defined by a grammar as
a set of transformation rules as stated above, is due to
\href{https://en.wikipedia.org/wiki/Noam_Chomsky}{Noam Chomsky}.

In more technical terms, the example of grammar above is called a
\href{https://en.wikipedia.org/wiki/Context-free_grammar}{\emph{context-free
grammar}} and plays an essential role in defining programming languages
and compilers, which translate a code into an executable.

\begin{itemize}
\tightlist
\item
  \textbf{Solved exercise:} Prove that the sentence
  \((\urcorner ((p\vee q)\wedge r))\) is a wff.

  \begin{itemize}
  \tightlist
  \item
    \textbf{Proof:}

    \begin{enumerate}
    \def\labelenumi{\arabic{enumi}.}
    \tightlist
    \item
      p, q and r are wff {[}by WFF1{]}
    \item
      \((p\vee q)\) is a wff {[}by 1 and WFF2{]}
    \item
      \(((p\vee q)\wedge r)\) is a wff {[}by 2 and WFF2{]}
    \item
      \((\urcorner ((p\vee q)\wedge r))\) is a wff {[}by 3 and WFF2{]} .
      q.e.d.
    \end{enumerate}
  \end{itemize}
\end{itemize}

\subsubsection{Semantics of Propositional
Logic}\label{semantics-of-propositional-logic}

The semantics of propositional logic is already well known by us: it
consists in specifying the truth tables of all connectives.

\section{Notes of Wed Nov. 7/8 2017.}\label{notes-of-wed-nov.-78-2017.}

\subsection{Parse trees and Truth
trees}\label{parse-trees-and-truth-trees}

For details, check out the book reference provided at the very beginning
of these notes.

\subsubsection{Parse Trees}\label{parse-trees}

Eg: Parse tree for \(\urcorner p\rightarrow (q \vee r)\)

\begin{equation}
\begin{array}{ccc}
\phantom{p} & \rightarrow & \phantom{p}
\\
\phantom{p} & \swarrow\,\searrow  & \phantom{p}
\\
\urcorner  & \phantom{\rightarrow} & \vee 
\\
\downarrow  & \phantom{\rightarrow} & \swarrow \searrow
\\
p & \phantom{\rightarrow} &  q \quad r 
\end{array}
\end{equation}

\subsubsection{Truth Trees}\label{truth-trees}

Advantages:

\begin{enumerate}
\def\labelenumi{\arabic{enumi}.}
\tightlist
\item
  \textbf{Simplifying} expressions/statements
\item
  Determining \textbf{SAT} (satisfiability) questions
\item
  Determining \emph{contradictions}/\textbf{tautologies}
\end{enumerate}

Eg: (Book notes: pp 51, example 1.10.1)

\begin{equation}
\begin{array}{ccc}
\phantom{p} & \urcorner p \rightarrow (q\wedge r) & \phantom{p}
\\
\phantom{p} & \swarrow\,\searrow  & \phantom{p}
\\
\urcorner (\urcorner p)  & \phantom{\rightarrow} & q\wedge r 
\\
\downarrow  & \phantom{\rightarrow} & \swarrow \searrow
\\
p & \phantom{\rightarrow} &  q \quad r 
\end{array}
\end{equation}

See book examples 1.10-2 to 1.10.10.

Homework: Exercices 5 (see book pp.~61)

\paragraph{Example workout: Exercise 1(a) book
pp.~61}\label{example-workout-exercise-1a-book-pp.61}

\begin{itemize}
\item
  \textbf{Problem}: Determine wether the following argument is valid or
  not using a truth tree:
  \(A\equiv p\rightarrow q,\,r\rightarrow s,\,p\vee r\vDash q\vee s\)

  \textbf{Solution}:

  The key is to see that proving such argument is valid is equivalent to
  proving that the following statement is a tautology

  \(\vDash\,\urcorner\left( (p\rightarrow q) \wedge (r\rightarrow s) \wedge (p\vee r)\right) \vee (q\vee s)\)

  The above line states that the expression to the right of the symbol
  \(\vDash\) is a tautology.

  In order to prove that an expression is a tautology the trick is to
  prove that its negation is a \textbf{contradiction} (\emph{proof by
  contradiction}), i.e., we need to prove that

  \(\left( (p\rightarrow q) \wedge (r\rightarrow s) \wedge (p\vee r)\right) \,\wedge \, \urcorner (q\vee s) \vDash .\)

  Let's do that.

  \begin{equation}
  \begin{array}{rcl}
  \phantom{p} & \left( (p\rightarrow q) \wedge (r\rightarrow s) \wedge (p\vee r)\right) \,\vee\, \urcorner (q\vee s) & \phantom{p} 
  \\
  \phantom{p} & \swarrow \qquad\qquad \qquad\qquad\qquad\searrow &  \phantom{p}
  \\
    \phantom{p} & (p\rightarrow q) \wedge (r\rightarrow s) \wedge (p\vee r) \quad\quad   \urcorner (q\vee s) & \phantom{a}
  \\
    \phantom{p} & \downarrow \qquad\qquad\qquad\qquad\qquad\qquad   \downarrow & \phantom{a}
  \\  
    \phantom{p} & p\rightarrow q \qquad\qquad\qquad\qquad\qquad   \urcorner q  & \phantom{a}
  \\
    \phantom{p} & r\rightarrow s \qquad\qquad\qquad\qquad\qquad   \urcorner s & \phantom{a}
  \\
    \phantom{p} & p\vee r        \qquad\qquad\qquad\qquad\qquad   \phantom{q\vee s} & \phantom{a}
  \\
    \phantom{p} & \swarrow\quad\searrow \quad\qquad\qquad\qquad\qquad\qquad   \phantom{q\vee s} & \phantom{a}
  \\
    \phantom{p} & \urcorner p\qquad\qquad q        \qquad \qquad\qquad\qquad\qquad\qquad  \phantom{q\vee s} & \phantom{a}
  \\
    \swarrow  & \downarrow \qquad \swarrow\quad\searrow        \qquad\qquad \qquad\qquad\qquad   \phantom{q\vee s} & \phantom{a}
  \\
    \urcorner r \quad  & s    \qquad  \urcorner r\qquad s \qquad\qquad\qquad\qquad\qquad \phantom{q\vee s} & \phantom{a}
  \\
    \swarrow\searrow  & \quad\swarrow\searrow   \quad  \downarrow\searrow \qquad \downarrow\searrow \qquad\qquad\qquad\qquad\qquad \phantom{q\vee s} & \phantom{a}
  \\
   \perp p\; \perp r  & \quad \perp p\quad r   \quad  p\; \perp r \quad p\quad r \qquad\qquad\qquad\qquad\qquad \phantom{q\vee s} & \phantom{a}
  \end{array}
  \end{equation}

  Clearly, not all branches end up in a contradiction, e.g., the
  right-most branch containing \(\urcorner q \wedge \urcorner s\).

  Whence, the original statement is what is called a \textbf{contingent
  statement}, i.e., \emph{a statement whose truth value can be either
  false or true, depending on the truth values of the atoms involved}.
  \(\square\)
\end{itemize}
