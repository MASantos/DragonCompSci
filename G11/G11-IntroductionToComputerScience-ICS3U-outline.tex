\documentclass{report}
\usepackage{hyperref}
\usepackage{graphicx}
\usepackage{float}
\usepackage{color}
\usepackage{listings}
\usepackage{enumitem}
\setlist[enumerate]{label*=\arabic*.}
\usepackage{datetime}
\newdate{date}{22}{06}{2017}
\date{\displaydate{date}}

\usepackage{titlesec}
\titleformat{\chapter}[display]
  {\normalfont\bfseries}{}{0pt}{\Huge}


\definecolor{lightgray}{rgb}{.9,.9,.9}
\definecolor{darkgray}{rgb}{.4,.4,.4}
\definecolor{purple}{rgb}{0.65, 0.12, 0.82}

\lstdefinelanguage{JavaScript}{
  keywords={break, case, catch, continue, debugger, default, delete, do, else, finally, for, function, if, in, instanceof, new, return, switch, this, throw, try, typeof, var, void, while, with},
  morecomment=[l]{//},
  morecomment=[s]{/*}{*/},
  morestring=[b]',
  morestring=[b]",
  sensitive=true
  ndkeywords={class, export, boolean, throw, implements, import, this},
  keywordstyle=\color{green}\bfseries,
  ndkeywordstyle=\color{darkgray}\bfseries,
  identifierstyle=\color{black},
  commentstyle=\color{purple}\ttfamily,
  stringstyle=\color{red}\ttfamily,
}

\lstset{language=JavaScript,
   backgroundcolor=\color{lightgray},
   extendedchars=true,
   %basicstyle=\footnotesize\ttfamily,
   basicstyle=\scriptsize\ttfamily,
   showstringspaces=false,
   showspaces=false,
   %numberstyle=\footnotesize,
   numberstyle=\scriptsize,
   numbersep=9pt,
   tabsize=2,
   breaklines=true,
   showtabs=false,
   captionpos=b
}

\title{Computer Science ICS4U G12 \\
The Dragon Academy \\
Course Outline
}

%\renewcommand{\contentsname}{Units}
%\renewcommand\thesection{{section}\arabic}
\setcounter{secnumdepth}{0}

\begin{document}
\maketitle
\tableofcontents
\chapter{Introduction}
\section{Overview}
This course outline follows the requirements set by the Ontario Ministry of Education as published 
in \href{http://www.edu.gov.on.ca/eng/curriculum/secondary/computer10to12_2008.pdf}{http://www.edu.gov.on.ca/eng/curriculum}

Computer Science ICS4U is a course that enables students to further develop knowledge and skills in computer
science. 

Students will use {\sl modular design principles} to create complex and {\sl fully documented programs}, 
according to industry standards. Student teams will manage a large
software development project, from planning through to project review. Students will
also {\sl analyse algorithms for effectiveness}. They will investigate ethical issues in comput-
ing and further explore environmental issues, emerging technologies, areas of research
in computer science, and careers in the field.

Prerequisite: Introduction to Computer Science, Grade 11, University Preparation (ICS3U)

\section{Expectations}
\newcommand{\expectation}[1]{(#1)}
\newcommand{\xpecA}[1]{ \expectation{\textcolor{red}{#1}} }
\newcommand{\xpecB}[1]{ \expectation{\textcolor{blue}{#1}} }
\newcommand{\xpecC}[1]{ \expectation{\textcolor{magenta}{#1}} }
\newcommand{\xpecD}[1]{ \expectation{\textcolor{green}{#1}} }
\begin{enumerate}
\item[\xpecA{A.}] {\bf Programming Concepts and Skills }
	\begin{enumerate}
	\item[A.1] {\bf Demonstrate ability to use different data types }
		\begin{enumerate}
		\item[A.1.1] Use integer division \& remainder 
		\item[A.1.2] Limitations of finite data ( integer bounds, float precision, rounding errors,...)
		\item[A.1.3] Use objects, structures/records
		\item[A.1.4] Use non-numeric comparison (strings, objects,...)
		\end{enumerate}
	\item[A.2] {\bf Describe and use modular programming concepts and principles }
		\begin{enumerate}
		\item[A.1.1] Divide program among multiple files
		\item[A.1.2] Reusable code techniques: encapsulation, inheritance, overloading, polymorphism,...
		\item[A.1.3] Modify existing code to enhance functionality
		\end{enumerate}
	\item[A.3] {\bf Design and write Algorithms to solve a variety of problems }
		\begin{enumerate}
		\item[A.3.1] I/O $<>$ file
		\item[A.3.2] Linear \& binary search
		\item[A.3.3] Insert/Delete array elements
		\item[A.3.4] Sort data in array
		\item[A.3.5] Algo for 2D arrays (vec, complex, matrix, images...)
		\item[A.3.6] Design simple \& efficient recursive algo ($n!$, merge-sort, parse XML,...)
		\end{enumerate}
	\item[A.4] {\bf Use proper code maintenance }
		\begin{enumerate}
		\item[A.4.1] Work independently
		\item[A.4.2] Implement testing
		\item[A.4.3] Document code
		\item[A.4.4] Document code and add external docs/guides
		\end{enumerate}
	\end{enumerate}
\item[\xpecB{B.}] {\bf Software Development }
	\begin{enumerate}
	\item[B.1] {\bf Project Management }
	\item[B.2] {\bf Project Contribution }
	\end{enumerate}
\item[\xpecC{C.}] {\bf Designing Modular Programs }
	\begin{enumerate}
	\item[C.1] {\bf Apply modular design concepts based on functionality and reusability ( modules, classes -w/ methods \& inheritance-, records/structures and ADT: stack, queue, dictionary,...) }
	\item[C.2] {\bf Analyse Algorithms for effectiveness in solving a problem ( Complexity -search, sort-, pitfalls of recursive algorithms -stack, exp growth) }
	\end{enumerate}
\item[\xpecD{D.}] {\bf Topics in Computer Science }
	\begin{enumerate}
	\item[C.1] {\bf Environmental Stewardship \& Sustainability }
	\item[C.2] {\bf Ethical practices }
	\item[C.3] {\bf Emerging technologies }
	\item[C.4] {\bf Exploring Computer Science }
		\begin{enumerate}
		\item[C.4.1] Interdisciplinarity CS $\leftrightarrow$ other-science (bioinformatics, linguistics, math,...)
		\item[C.4.2] Topics in Theoretical CS (crypto, graphs, logic, computability, data mining, AI, ...)
		\end{enumerate}
	\end{enumerate}
\end{enumerate}
\chapter{Units}
\section{Unit 1: Computers, the Internet and Society \xpecD{D1, D2, D3, D4}}
\begin{enumerate}
\item The computational revolution
	\begin{enumerate}
	\item Earliest calculating devices
	\item From Electricity to Electronics: The revolution of the Transistor
	\item The future revolutions: Quantum Computers, Biological Computers, AI...what next?
	\end{enumerate}
\item The networking revolution, aka. Internet
	\begin{enumerate}
	\item The Information flow and Society.
	\item The Internet OF Things (IoT)
	\end{enumerate}
\item Life Cycle Analysis: The resources demand of it all
	\begin{enumerate}
	\item What is a Life Cycle Analysis of a product
	\item Are computers (environmentally) clean? Is the {\sl cloud} clean?
	\item Assignment: LCA of a computer
	\begin{enumerate}
	\item How much does rough materials extraction pollute? How much energy does it consume?
	\item How much does manufacturing pollute?  How much energy does it consume?
	\item How much does computer pollute?  How much energy does it consume?
	\item How much does disposing of a computer pollute? How much energy does it consume?
	\item Assignment Resources:
		\begin{enumerate}
		\item \url{http://homes.soic.indiana.edu/nensmeng/posts/2013/05/20/computers-environment/}
		\item \href{http://homes.soic.indiana.edu/nensmeng/enviro-compute/}{Dirty Bits: An environmental History of Computing}
		\item \url{http://www.isc.uoguelph.ca/documents/050602environcs.pdf}
		\item Google: Power consumption of data centers
		\item Google: Environmental impact of manufacturing computer
		\end{enumerate}
	\end{enumerate}
	\end{enumerate}
\item Carriers in Computing
\end{enumerate}
\section{Unit 2: How Computers work: Hardware and Software \xpecA{A1, A2, A3, }}
\begin{enumerate}
\item Number Systems
	\begin{enumerate}
	\item Integrated Circuits and Transistors: The 0's and 1's
	\item Finite representation of integers and fractionary numbers: The decimal, binary, hexadecimal and sexagesimal number systems
	\item Floating point representation of real numbers
	\end{enumerate}
\end{enumerate}
\section{Unit 3: The Logic of it all \xpecA{A1, A3}}
\begin{enumerate}
\item Logic Gates and Logic Circuits
	\begin{enumerate}
	\item Reasoning with Logic: The Truth Values
	\item Logic gates as words: AND, OR, NOT.
	\item Identify and sketch the diagram of each of the three basic logic gates. Label all inputs and outputs.
	\item Write the truth table of each basic logic gate
	\item Analyze and write the truth table of logic circuits
	\end{enumerate}
\item What a computer is in a nutshell
	\begin{enumerate}
	\item Von Neumann Architecture: The essential diagram of a computer.
	\item Turing Machine: The essential mechanics of a computer 
	\item The Busy Beaver: Given the rules of a TM, an input string of 0s and 1s and an initial state,
	        determine if the TM will halt or not and what will be on the tape in the end.
	\end{enumerate}
\item Formal demonstrations
	\begin{enumerate}
	\item Propositional Logic
	\item The language of First-Order Logic
	\end{enumerate}
\end{enumerate}
\section{Unit 4: Object-Oriented Programming \xpecA{A1,A2,A3,A4},\xpecC{C1}, \xpecD{D3, D4}}
\begin{enumerate}
\item JavaScript: The prototypal approach to OOP
\end{enumerate}
\section{Unit 5: Algorithms and Information \xpecA{A3, A4}, \xpecB{B1, B2}, \xpecC{C1,C2}, \xpecD{D3, D4}}
\begin{enumerate}
\item Compression
\item Searching: unsorted, sorted and hashed lists
\item Information Theory
	\begin{enumerate}
	\item Decision Tree: Number of yes/no questions needed
	\item Information as \# yes/no questions needed
	\item Information, bits and integer logarithms
	\end{enumerate}
\end{enumerate}
\end{document}
