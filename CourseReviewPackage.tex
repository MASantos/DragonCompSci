\documentclass{article}
\usepackage{graphicx}
\usepackage{float}
\usepackage{color}
\usepackage{listings}

\definecolor{lightgray}{rgb}{.9,.9,.9}
\definecolor{darkgray}{rgb}{.4,.4,.4}
\definecolor{purple}{rgb}{0.65, 0.12, 0.82}

\lstdefinelanguage{JavaScript}{
  keywords={break, case, catch, continue, debugger, default, delete, do, else, finally, for, function, if, in, instanceof, new, return, switch, this, throw, try, typeof, var, void, while, with},
  morecomment=[l]{//},
  morecomment=[s]{/*}{*/},
  morestring=[b]',
  morestring=[b]",
  sensitive=true
  ndkeywords={class, export, boolean, throw, implements, import, this},
  keywordstyle=\color{green}\bfseries,
  ndkeywordstyle=\color{darkgray}\bfseries,
  identifierstyle=\color{black},
  commentstyle=\color{purple}\ttfamily,
  stringstyle=\color{red}\ttfamily,
}

\lstset{language=JavaScript,
   backgroundcolor=\color{lightgray},
   extendedchars=true,
   %basicstyle=\footnotesize\ttfamily,
   basicstyle=\scriptsize\ttfamily,
   showstringspaces=false,
   showspaces=false,
   %numberstyle=\footnotesize,
   numberstyle=\scriptsize,
   numbersep=9pt,
   tabsize=2,
   breaklines=true,
   showtabs=false,
   captionpos=b
}

\title{Exploring Computer Technology TEJ0 G9 \\
The Dragon Academy \\
Course Review }

\begin{document}
\maketitle


\section{Index of topics}
\begin{enumerate}
\item[ 1.] Turing Machines:

	\begin{enumerate}
	\item[ 1.1.] Given the rules of a TM, an input string of 0s and 1s and an initial state,
	        determine if the TM will halt or not and what will be on the tape in the end.
	\end{enumerate}
\item[ 2.] Logic Gates and Logic Circuits
	\begin{enumerate}
	\item[ 2.1.] Truth values
	\item[ 2.2.] Logic gates as words: AND, OR, NOT.
	\item[ 2.3.] Identify and sketch the diagram of each of the three basic logic gates. Label all inputs and outputs.
	\item[ 2.4.] Write the truth table of each basic logic gate
	\item[ 2.5.] Analyze and write the truth table of logic circuits
	\end{enumerate}
\item[ 3.] Binary Numbers
	\begin{enumerate}
	\item[ 3.1.] Given a binary number of 8 bits, determine its decimal representation
	\item[ 3.2.] Given an integer in the decimal number system between 0 and 1024,
	     write its binary representation.
	\end{enumerate}
\item[4.] Web pages: HTML and CSS
	\begin{enumerate}
	\item[4.1] HTML
		\begin{enumerate}
		\item[4.1.1] The template of a web page: the {\sl doctype, html, head, title} and {\sl meta} body tags
		\item[4.1.2] Headings: h1, h2, and h3 
		\item[4.1.3] Paragraphs: p
		\item[4.1.4] Boldface and emphasize: strong and em.
		\item[4.1.5] Inserting an image: img
		\item[4.1.6] Inserting a link: a
		\item[4.1.7] The box tag: div
		\end{enumerate}
	\item[4.2] CSS: the style tag
		\begin{enumerate}
		\item[4.2.1] Changing the font color of an HTML element: \mbox{color}
		\item[4.2.2] Changing the background color of an HTML element: \mbox{background}
		\item[4.2.3] Changing the font type (family) of an HTML element: \mbox{font-family}
		\item[4.2.4] Changing the font size of an HTML element: \mbox{font-size}
		\item[4.2.5] Changing the border of an HTML element: \mbox{border}
		\item[4.2.6] Animations (not an exam topic)
		\end{enumerate}
	\end{enumerate}
\item[ 5.] Javascript programing
	\begin{enumerate}
	\item[ 5.1.] Assignments
		\begin{enumerate}
             	\item[ 5.1.1.] Assigning numbers to a variable
		\item[ 5.1.2.] Assigning strings to a variable
		\end{enumerate}
	\item[ 5.2.] Inputs
		\begin{enumerate}
            	\item[ 5.2.1.] Inputting numbers from the user
            	\item[ 5.2.2.] Inputting strings from the user
		\end{enumerate}
	\item[ 5.3.] Loops
		\begin{enumerate}
            	\item[ 5.3.1.] Calculate the sum of the first 100 integers
            	\item[ 5.3.2.] Calculate the product of the first 20 integers
            	\item[ 5.3.3.] General problem: Identify the pattern and implement the while-loop
		\end{enumerate}
	\end{enumerate}
\item[ 6.] Algorithms
	\begin{enumerate}
	\item[6.1.] Compression
	\item[6.1] Searching: unsorted, sorted and hashed lists
	\item[6.2] Information Theory
		\begin{enumerate}
     		\item[ 6.2.1.] Decision Tree: Number of yes/no questions needed
     		\item[ 6.2.2.] Information as \# yes/no questions needed
		\item[6.2.3.] Information, bits and integer logarithms
		\end{enumerate}
	\end{enumerate}
\end{enumerate}
     
\section{Examples}
\subsection{1. Turing Machine}
\begin{figure}[H]
\includegraphics[width=0.8\linewidth]{IMG_20180608_153248.jpg}
\end{figure}
\begin{figure}[H]
\includegraphics[width=0.8\linewidth]{IMG_20180531_111229.jpg}
\end{figure}
\begin{figure}[H]
\includegraphics[width=0.8\linewidth]{IMG_20180531_111238.jpg}
\end{figure}

\subsection{2. Logic Gates and Circuits}
\begin{figure}[H]
\includegraphics[width=0.8\linewidth]{IMG_20180607_112038.jpg}
\end{figure}
\begin{figure}[H]
\includegraphics[width=0.8\linewidth]{IMG_20180607_112028.jpg}
\end{figure}
\begin{figure}[H]
\includegraphics[width=0.8\linewidth]{IMG_20180607_111900.jpg}
\end{figure}
\begin{figure}[H]
\includegraphics[width=0.8\linewidth]{IMG_20180607_111758.jpg}
\end{figure}
\begin{figure}[H]
\includegraphics[width=0.8\linewidth]{IMG_20180607_111758.jpg}
\end{figure}
\begin{figure}[H]
\includegraphics[width=0.8\linewidth]{IMG_20180607_111747.jpg}
\end{figure}
\begin{figure}[H]
\includegraphics[width=0.8\linewidth]{IMG_20180607_111735.jpg}
\end{figure}
\begin{figure}[H]
\includegraphics[width=0.8\linewidth]{IMG_20180607_111725.jpg}
\end{figure}

\subsection{3. Binary Numbers}
In a group of 8 bits, the position of each of them represents a
different value. For instance, if we set all to $0$ (switched off)
except the right-most bit, which we set to $1$ (switched on), the
value represented is 1.

If instead the only bit we switch on is the second right-most one, then
the value we have is 2. And so on and so forth.
\begin{eqnarray*}
0000\,0001 &=& 1 \\
0000\,0010 &=& 2 \\
0000\,0100 &=& 4 \\
0000\,1000 &=& 8 \\
0001\,0000 &=& 16 \\
0010\,0000 &=& 32\\
0100\,0000 &=& 64\\
1000\,0000 &=& 128 
\end{eqnarray*}

When we have several bits switched on, we just "collect" the values represented by each
of them and add them up
\begin{eqnarray*}
0101\,0101\,=\,64\,+\,16\,+\,4\,+\,1\,=\,85 \\
1000\,1010\,=\,128\,+\,8\,+\,2\,=\,138
\end{eqnarray*}

\begin{figure}[H]
\includegraphics[width=0.8\linewidth]{IMG_20180611_103420.jpg}
\end{figure}

We can also do the other way around: given a decimal number like $35$ write its binary representation.

Solution: First try to write $35$ in terms of the previous powers of 2. $35\,=\,32+3\,=\,32+2+1$.
Hence, $35$ in binary is $0010\,0011$.

What is $127$ in binary? 

Solution: $127\,=\,64\,+\,32\,+\,16\,+\,8\,+\,4\,+\,2\,+\,1$, hence in binary $0111\,1111$.

What is $255$ in binary? Check that the solution is all 8 bits set: $1111\,1111$

What numbers are the following?
\begin{eqnarray*}
0000\,0011 &=& \\
0000\,0111 &=& \\
0000\,1111 &=& \\
0001\,1111 &=& \\
0011\,1111 &=& \\
0111\,1111 &=& \\
\end{eqnarray*}

Solution: $3,7,15,31,63,127$

\subsubsection{More than 8 bits}
\begin{figure}[H]
\includegraphics[width=0.8\linewidth]{IMG_20180611_103430.jpg}
\end{figure}
\begin{eqnarray*}
0001\,0000\,0000 &=& 256 \\
0010\,0000\,0000 &=& 512 \\
0100\,0000\,0000 &=& 1024 \\
1000\,0000\,0000 &=& 2048 \\
\end{eqnarray*}

What is the binary expression of $725$? 

Solution: We proceed as before, expressing this number as a sum of powers of 2,
namely, $725\,=\,512\,+\,128\,+\,64\,+\,16\,+\,4\,+\,1$, hence in binary it is
$0010\,1101\,0101$.

\subsection{4. Web pages}
\begin{figure}[H]
\includegraphics[width=0.8\linewidth]{IMG_20180608_153314.jpg}
\end{figure}
\begin{figure}[H]
\includegraphics[width=0.8\linewidth]{IMG_20180608_153324.jpg}
\end{figure}
\begin{figure}[H]
\includegraphics[width=0.8\linewidth]{IMG_20180608_153335.jpg}
\end{figure}
\begin{figure}[H]
\includegraphics[width=0.8\linewidth]{IMG_20180608_153346.jpg}
\end{figure}

\subsection{5. Programing in Javascript}
\begin{figure}[H]
\includegraphics[width=0.8\linewidth]{IMG_20180611_103440.jpg}
\end{figure}
\begin{lstlisting}
var x = 3.5
var y = 7
var z = x*y
var s = Math.pow(x,2)
alert( z )
\end{lstlisting}
The variable x gets assigned the value $3.5$, y $7$ and z the product of those two,
namely $24.5$.

We can calculate the square of the value contained in x by writing "Math.pow(x,2)". 
There is a second method though, namely simply writing the product of x by itself: "x*x".

"alert" is a way to output information to the user. It creates a new small pop-up window showing
the content of whatever we put between the parentheses after alert. In this case it will show the
value of z.

We can 
\begin{figure}[H]
\includegraphics[width=0.8\linewidth]{IMG_20180611_103448.jpg}
\end{figure}
In a similar way we can deal with "strings":
\begin{lstlisting}
var name = "George"
var familyName = "Lucas"
var fullName = name + " " + familyName
alert("This person's name is "+ fullName)
\end{lstlisting}
We can compose strings out of smaller strings by using the plus $+$ sign. 


\begin{figure}[H]
\includegraphics[width=0.8\linewidth]{IMG_20180611_103504.jpg}
\end{figure}
\begin{figure}[H]
\includegraphics[width=0.8\linewidth]{IMG_20180611_103514.jpg}
\end{figure}
\begin{figure}[H]
\includegraphics[width=0.8\linewidth]{IMG_20180611_103523.jpg}
\end{figure}
We can ask the user for input through the command "prompt". If we expect a number, we
need to wrap the prompt with the function "Number" 
\begin{lstlisting}
var user_name = prompt("Enter your name:")

var user_number = Number( prompt("Enter a number:")  )

alert( "The user "+ user_name + " entered the value "+ user_number )
\end{lstlisting}


\begin{figure}[H]
\includegraphics[width=0.8\linewidth]{IMG_20180611_103534.jpg}
\end{figure}
Loops allow repetition of statements without actually having to write them several times.
Loops allow use to  calculate tedius mathematical expressions.

Write the sum of the first 100 numbers. Store the result into a variable called "sum" 
and show it to the user.
\begin{lstlisting}
var sum = 0
var i = 1 
while( i <= 100 ){
    sum = sum + i
    i = i + 1
}
alert( "The sum of the first 100 integers is "+ sum )
\end{lstlisting}

\subsection{6. Algorithms}
\begin{figure}[H]
\includegraphics[width=0.8\linewidth]{IMG_20180529_111617.jpg}
\end{figure}
\begin{figure}[H]
\includegraphics[width=0.8\linewidth]{IMG_20180529_111640.jpg}
\end{figure}

\end{document}
