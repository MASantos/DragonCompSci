\documentclass{article}
\usepackage{amssymb}
\usepackage{circuitikz}
\usepackage[T1]{fontenc}
\usepackage[colorlinks=true,urlcolor=blue]{hyperref}


\begin{document}
\title{Computer Science G12}
\author{Term 1, Review Test \\Date: Thu. 9 November 2017 \\  \\{\bf Name:}\\ }
\date{}
\maketitle

%\section{Problems (100\%)
%}
%\subsection{Logic Gates}

%Note: For the questions on Logic Circuits, remember to label each input and output lines and to use these labels correctly on the truth tables.

\begin{enumerate}
\item ($30\%$) Answer the following questions: 
\begin{enumerate}
\item  Complete the table by converting the values to the different base systems. 

\begin{math}
\begin{array}{|c|c|c|}
Binary & Hex & Dec \\
\hline
1\,1011\,1110\,1110\,1101 & \; & \; \\
\hline
\; & FEAD & \; \\
\hline
\; & \; & 65537 \\
\hline
\; & \; & 25/8 \\
\hline
\; & \; & 71/3 \\
\hline
\end{array}
\end{math}
\item Convert the numbers -37 and -256 into binary using 2-complement. Make sure your answer has a bit-length that is multiple of a byte.
\item Can the above fractional values be expressed exactly in those base systems? If not, find the smallest two bases where that's possible.
\end{enumerate}
\item ($10\%$) What follows are the transition rules of a Turing Machine. Write down the computations steps when this machines starts at state and input given by the first line below. Does this TM halt for this "program"?

\begin{math}
\begin{array}{rcl}
(A,1) & \longrightarrow & (1,C,\rightarrow) \\
(A,0) & \longrightarrow & (0,A,\leftarrow) \\
(C,0) & \longrightarrow & (1,A,\rightarrow) \\
(C,1) & \longrightarrow & (0,D,\leftarrow)  \\
(D,1) & \longrightarrow & (0,A,\leftarrow) \\
(D,0) & \longrightarrow & (0,C,\rightarrow) \\
\end{array}
\end{math}

\begin{math}
\begin{array}{l|r}
\hline \\
A & 1.\,0\;0\;0\;1\;1\;E \\
\hline
\phantom{A} & \phantom{0}
\end{array}
\end{math}

\vspace{16cm}
\item ($20\%$)  Answer the following questions:
\begin{enumerate}
\item From $p\rightarrow q$ and $q\rightarrow p$, conclude $\urcorner p\,\wedge\,\urcorner q\,\vee\,q\,\wedge\, p$ using a Truth Table.
\item Idem, but prove it algebraically.
\end{enumerate}
\vspace{16cm}
\item ($10\%$) Prove that the expression $(\urcorner ((p\vee q)\wedge r))$ is a {\sl wff} in propositional logic.
\vspace{4cm}
\item ($30\%$) We defined the symbol $\vDash$ to denote {\sl arguments} and we said that the argument $A\vDash B$ is valid {\bf iff} $\vDash A\rightarrow B$, 
that is, we can rewrite the argument $A\vDash B$ as the statement "$A\rightarrow B$ is a tautology".
Consider the expression $A\equiv p\rightarrow q,\,r\rightarrow s,\,p\vee r\vDash q\vee s$.
\begin{enumerate}
\item For any given expression $X$, if we state that $\vDash X$, what can we say about $\urcorner X$? How can we express this using the symbol $\vDash$?
\item Rewrite expression $A$ as a tautology {\sl argument}.
\item Determine wether the argument $A$ is or not valid using truth trees.
\end{enumerate}

\end{enumerate}


\end{document}
# Computer Science G12 

## Boolean Logic


### Assignment 4. Due date: Fri. 27 Oct. 2017

See examples below. Note p',!p or ~p, all mean NOT p.

1. Write the truth table for the biconditional: p <-> q
1. Prove p = p.q + p.(~q)
1. From p->q and q->p, conclude p'.q'+qp using Truth Table
1. Idem, but prove it algebraically
1. Find equivalent expression w/o any conditional nor biconditional : (q<->p) + (p=>q)
1. Prove using TT: (p->q) (~q) -> p
1. Idem, algebraically
1. Prove using TT: X.(X+Y) = X
1. Idem, algebraically
1. Write the dual expression of the following ones:
   1. p.(!p+q)
   1. (X+Y).(X+!Y).(!X+Y)
   1. (A.1)+(A+0+~A)
   1. (A+B).(~A+B)
   1. ABC+ A(!B)C + (!A)B(!C)
1. From p->q and q->p, conclude algebraically q'+qp 
1. From p->q and q->p, conclude algebraically p'+qp 
1. Find the complement of x.(y'.z'+y.z)
1. Simplify using the laws of Boolean algebra: x.y+x'.z+y'.z


**Example:** Let's prove algebraically that, given p->q and q->p, we can conclude ~p+qp.
  Remember that "=>" means "from the expression on the left we can conclude the expression on the right", and viceversa for the arrow in the other sense.
  
  From the statement of the problem we assume we can claim

     1. (p->q).(q->p)  (problem assumption)
     2. (p->q)(from 1 and definition of conjunction)
     3. ~p+q  (from 2. and definition of the conditional operator)
     4. ~{p.(~q)} (from 3 and De Morgan laws)
     5. ~{[p.(~q)] + F} (from 4 and definition of conjunction and definition of F)
     6. ~{[p.(~q)] + (~p.p)} (from 5 and definition of conjunction )
     7. ~{p.[(~q)+(~p)]} (from 5 and distributive laws; also from 5 in CNF )
     8. ~p+qp ( from 6 and de Morgan Laws)
     q.e.d

  The sequence of steps above means that we claim to be true that "step i => step i+1", and we argue so for the reason in parenthesis given on the right of each step.

  Another way to prove it is starting from the 7 and reaching up to 2. From there to 1 requires only the additional assumption of (q->p) which is given in the problem statement. This would finish this other proof.


